%&latex
% Derived from `samplecards.tex',

\documentclass[avery5388,grid,frame]{flashcards}

\cardfrontstyle[\large\slshape]{headings}
\cardbackstyle{empty}

\begin{document}

\cardfrontfoot{Scientific Computing}

\begin{flashcard}[Establishing a remote connection to the crc]{ssh netid@crcfe01.crc.nd.edu

\\ ssh netid@crcfeIB01.crc.nd.edu}

  A real-valued function $||x||$ defined on a linear space $X$, where
  $x \in X$, is said to be a \emph{norm on} $X$ if

  \smallskip

  \begin{description}
    \item [Positivity]            $||x|| \geq 0$,
    \item [Triangle Inequality]   $||x+y|| \leq ||x|| + ||y||$,
    \item [Homogeneity]           $||\alpha x|| = |\alpha| \:  ||x||$,
                                  $\alpha$ an arbitrary scalar,
    \item [Positive Definiteness] $||x|| = 0$ if and only if $x=0$,
  \end{description}

  \smallskip

  where $x$ and $y$ are arbitrary points in $X$.

  \medskip

  A linear/vector space with a norm is called a \emph{normed space}.
\end{flashcard}

\begin{flashcard}[Definition]{Inner Product}

  Let $X$ be a complex linear space. An \emph{inner product} on $X$ is
  a mapping that associates to each pair of vectors $x$, $y$ a scalar,
  denoted $(x,y)$, that satisfies the following properties:

  \medskip

  \begin{description}
    \item [Additivity]            $(x+y,z) = (x,z) + (y,z)$,
    \item [Homogeneity]           $(\alpha \: x, y) = \alpha (x,y)$,
    \item [Symmetry]              $(x,y) = \overline{(y,x)}$,
    \item [Positive Definiteness] $(x,x) > 0$, when $x\neq0$.
  \end{description}
\end{flashcard}

\begin{flashcard}[Definition]{Linear Transformation/Operator}

  A transformation $L$ of (operator on) a linear space $X$ into a linear
  space $Y$, where $X$ and $Y$ have the same scalar field, is said to be
  a \emph{linear transformation (operator)} if

  \medskip

  \begin{enumerate}
    \item $L(\alpha x) = \alpha L(x), \forall x\in X$ and $\forall$
          scalars $\alpha$, and
    \item $L(x_1 + x_2) = L(x_1) + L(x_2)$ for all $x_1,x_2 \in X$.
  \end{enumerate}

\end{flashcard}

\end{document}

\endinput
%%
%% End of file `samplecards.tex'.

