%

\documentclass[a4paper]{article}

\renewcommand{\thesubsection}{\thesection.\alph{subsection}}

\usepackage[english]{babel}
\usepackage[utf8]{inputenc}
\usepackage{amsmath,amssymb,amsthm,mathrsfs,amsfonts,dsfont} 
\usepackage{graphicx}
\usepackage[colorinlistoftodos]{todonotes}

\usepackage{chngcntr}

\usepackage{mathtools}
\counterwithin{equation}{section}

\def\e{\vec{e}}
\def\a{\vec{a}}
\def\b{\vec{b}}
\def\c{\vec{c}}
\def\u{\vec{u}}
\def\v{\vec{v}}
\def\w{\vec{w}}
\def\EI{\vec{e}^{\:i}}
\def\EJ{\vec{e}^{\:j}}
\def\EK{\vec{e}^{\:k}}
\def\EL{\vec{e}^{\:l}}
\def\ei{\vec{e}_{i}}
\def\ej{\vec{e}_{j}}
\def\ek{\vec{e}_{k}}
\def\el{\vec{e}_{l}}

\title{\large AME 60624: Continuum Mechanics - Homework 2 \\ Prof. Karel Matou\v{s} (T.A.: Cale Harnish)}

\author{Oyekola Oyekole}

\date{\today}

\begin{document}

%\maketitle

% \begin{abstract}
% Here is the abstract
% \end{abstract}

\section{Minimum number of terms (alternating test)}
\label{problem1}
The error in a decreasing alternating series is bounded by the next term in the series. 
For a series \begin{align*}
S = \sum_{k=0}^\infty (-1)^ka_k \enspace,
\end{align*}
the minimum number of terms ($N$) needed to be certain of accuracy within a chosen tolerance $\epsilon$ (positive)\ is 
\begin{align}
|a_{N+1}|\le \epsilon \enspace.
\end{align}
Here, we have 
\begin{align*}
S = \sum_{k=0}^\infty(-1)^k\frac{1}{2k+1+\ln(k^2+1)}
\end{align*}
so we need to find $N$ such that 
\begin{align}
|a_{N+1}|=\left|\frac{1}{2N+1+\ln(N^2+1)}\right| \le \epsilon \quad .
\end{align}
Since $N > 0$ then $\ln(N^2+1)$ is positive, and the denominator is positive. So we can say
\begin{align}
\frac{1}{2N+1+\ln(N^2+1)} &\le \epsilon \notag \\
\implies 2N+1+\ln(N^2+1) &\ge \frac{1}{\epsilon}
\end{align}
Now, $x \ge \ln(x+1)$ when $x > -1$, so $N^2 \ge \ln(N^2 + 1)$ for $N^2 > 0$. Therefore,
\begin{align}
2N+1+N^2 &\ge \frac1{\epsilon} \notag\\
(N+1)^2 &\ge \frac{1}{\epsilon} \notag\\
N &\ge\sqrt{\frac1{\epsilon}}-1
\end{align}
\end{document}
