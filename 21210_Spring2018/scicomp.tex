%&latex
%% Derived from: asaetr.tex v1.0 01 Jan 92

%\documentclass{asaetr} % footnote - \itshape is invalid in math mode
\documentstyle{asaetr}
% requires cmcscsl10 font (Slanted Small Caps).
% 
\usepackage{hyperref}
\usepackage{xcolor}
\newcommand{\urlwofont}[1]{\urlstyle{same}\url{#1}}

\title{Quick Reference / Guide\thanks{For Spring 2017 ACMS 21210 Scientific Computing Lab}
      }
\author{ACMS 21210 \thanks{Applied \& Computational Math \& Statistics, University of Notre Dame IN 46556.}
       %\student
       %\and
       %A.~D.~Whittaker \member
       }
       % the format is: name \membership_grade, where membership_grade
       % is one of ( \member, \associate, \student, \affiliate, \fellow)

\begin{document}

\bibliographystyle{asaetr}
\maketitle

\begin{abstract}
A guide for quick reference in our scientific computing labs.\end{abstract}

\section{Remote Connectivity}


Windows SSH\ clients: \begin{itemize}
\item 
{\bf Bitvise}: \scriptsize\url{https://www.bitvise.com/ssh-client-download}
\item {\bf PuTTY}: \scriptsize\url{http://www.chiark.greenend.org.uk/\~sgtatham/putty/latest.html}
\item{\bf  WinSCP}: \scriptsize\url{https://winscp.net/eng/download.php}
\end{itemize}
\\
Mac SSH client: use the operating system's {\bf Terminal.app}

\subsection{Establishing a Remote Connection}

{\tt ssh USERNAME@SERVER -p PORT}
\\ where server = {\bf server IP\ address} or {\bf domain}. Default port is 22, so it can be omitted in some cases 
\\ e.g. {\tt ssh johndoe@crcfe01.crc.nd.edu}
\\ You will be required to enter your {\bf password}.
\\ The crcfeIB01 a.k.a the {\it infini-band} server is faster but you can only connect to it when on campus or using ND's VPN
\\ e.g. {\tt ssh johndoe@crcfeIB01.crc.nd.edu}

\subsection{REMOTE FILE MANAGEMENT}

Available for both Windows and Mac
\begin{enumerate}
\item Cyberduck client: {\scriptsize\url{https://cyberduck.io/}} 
\item Filezilla client: {\scriptsize\url{https://filezilla-project.org/}}
\end{enumerate}

\subsection{Some Common Linux Commands}
\begin{description}
\item[\tt whoami] {\it Displays who you are logged in as}
\item[\tt ls] {\it Directory listing}
\item[\tt ls -la] {\it Formatted listing (including hidden files)}
\item[\tt pwd] {\it Show the current working directory}
\item[\tt cd] {\it Go to home directory}
\item[\tt{cd PATH}] {\it Go to the path e.g. \tt{cd Public/code}}
\item[\tt{cd ..}] {\it Go to the upper folder}
\item[\tt{touch FILENAME}] {\it Create a file named FILENAME}
\item[\tt{mkdir DIRNAME}] {\it Create a folder named DIRNAME}
\item[\tt{rm FILENAME}] {\it Delete a file named FILENAME}
\item[\tt{rm -f DIRNAME}] {\it Force delete a file named FILENAME}
\item[\tt{rm -r DIRNAME}] {\it Delete a folder named DIRNAME}
\item[\tt{rm -fr DIRNAME}] {\it Force delete a folder (extreme caution!)}
\item[\tt{cp FILE1 FILE2}] {\it Copy FILE1 to FILE2}
\item[\tt{cp DIR1 DIR2}] {\it Copy DIR1 to DIR2}
\item[\tt{cp -r DIR1 DIR2}] {\it Copy DIR1 to DIR2, create if necessary}
\item[\tt{mv FILE1 FILE2}] {\it Move FILE1 to FILE2 (also for renaming)}
\item[\tt{mv FILE1 DIR2}] {\it Move FILE1 into existing folder DIR2}
\item[\tt{ln -s FILE LINK}] {\it Create a symbolic LINK to a FILE}
\item[\tt{nano FILE}] {\it Edit FILE using the nano text editor}
\item[\tt{head FILE}] {\it Show first 10 lines of FILE}
\item[\tt{tail FILE}] {\it Show last 10 lines of FILE}
\item[\tt{wget URL}] {\it Download resource from a LINK}
\end{description}

\subsection{Compiling your C++ code on the CRC}
Supply the file name of the source code, version of the C++ standard, and (optional, defaults to {\tt a.out}) the name of the output file.
\\
{\tt g++ SOURCE.cpp -std=c++11 -o RESULT.out}

\subsection{Running your output file on the CRC}

{\tt ./RESULT.out}

\subsection{Modifying your profile on the CRC}
({\it An advanced, optional setup})\\
{\tt cd}
\\ {\tt nano .cshrc}
\\ Add the lines to customize your environment, e.g.
\\ {\tt alias ls "ls --color=always"}
\\ {\tt alias scicomp "cd \sim\textnormal{/Public/code/21210"}}
\\ {\tt bindkey -k up history-search-backward}
\\ {\tt bindkey -k down history-search-forward}
\\\textnormal{{\it Save and exit}}

\end{document}


